\documentclass[10pt, oneside]{article} 
\usepackage{amsmath, amsthm, amssymb, calrsfs, wasysym, verbatim, bbm, color, graphics, geometry}

\geometry{tmargin=.75in, bmargin=.75in, lmargin=.75in, rmargin = .75in}  

\newcommand{\R}{\mathbb{R}}
\newcommand{\C}{\mathbb{C}}
\newcommand{\Z}{\mathbb{Z}}
\newcommand{\N}{\mathbb{N}}
\newcommand{\Q}{\mathbb{Q}}
\newcommand{\Cdot}{\boldsymbol{\cdot}}

\newtheorem{thm}{Theorem}
\newtheorem{defn}{Definition}
\newtheorem{conv}{Convention}
\newtheorem{rem}{Remark}
\newtheorem{lem}{Lemma}
\newtheorem{cor}{Corollary}
\newtheorem{example}{Example}
\newtheorem{exe}{Exercise}

\title{Some Identities}
\author{[Drew Remmenga]}


\begin{document}

\maketitle

\vspace{.25in}
\begin{align}
    lim_{x\to \infty}\frac{B_{2n}}{B_{2n+2}}=\pi^{2}
\end{align}
\begin{align}
    \pi = \frac{4}{1+\frac{1}{3+\frac{2}{5+\frac{3}{7+\frac{4}{\dots}}}}}
\end{align}
If s(n) = pentagonal numbers such that $s(n)= \frac{3n^{2}+n}{2}$ the nth partition number is given by 
\begin{align*}
    p(n) = \sum_{i=1}^{n+1}\sum_{k=1}^{i+1}\sum_{\substack{t=s(-k)\\\text{step }s(k)-s(-k)}}^{s(k)}\left\lceil\frac{sgn(i-t)+1}{2}\right\rceil(-1)^{k+1}p(\left\lceil\frac{sgn(i-t)+1}{2}\right\rceil)*(n-t)
\end{align*}
If p(0) = 1.
\begin{align*}
    -1+\sqrt{3}=\frac{1}{1+\frac{1}{2+\frac{1}{1+\frac{1}{2+\frac{1}{\dots}}}}}
\end{align*}
\begin{align*}
    \frac{\sqrt{2}}{2} = \frac{\pi}{4}\prod_{n=1}^{\infty}(1+\frac{4}{n})\prod_{n=-1}^{-\infty}(1-\frac{4}{n})
\end{align*}   
\end{document}